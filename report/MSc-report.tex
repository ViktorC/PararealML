\documentclass{article}

\usepackage{epcc}
\usepackage{graphics}
\usepackage{booktabs}

\renewcommand{\labelenumii}{\theenumii}
\renewcommand{\theenumii}{\theenumi.\arabic{enumii}.}


\begin{document}

\pagenumbering{roman}

\title{Parallel-in-time Methods with Machine Learning - Report}
\author{Viktor Csomor}
\date{\today}

\makeEPCCtitle

\newpage

\tableofcontents

\newpage
\pagenumbering{arabic}

\section{Introduction}

Differential equations can model a wide array of dynamical systems that play important roles in science, engineering, and finance. Examples of these include population growth, the motion of fluids, and changes in stock prices. While some simple differential equations can be solved analytically, most of them require a numerical approach \cite[p.~310]{suli2003}. Due to the importance of these models, their accurate and efficient numerical solution is a well researched area of numerical analysis. A lot of this research is focused on the parallelisation of numerical solvers to achieve better accuracy or reduce execution times. Traditionally, this parallelisation is applied either across the system or across time \cite{gear1988}.

\section{Background and Literature Review}

To do...

\section{Preliminary Investigations}



\section{Final Proposal}



\section{Workplan}

\begin{enumerate}
    \item Implement framework
        \begin{enumerate}
        	\item Add support for systems of ODEs
        	\item Add support for PDEs
        	\item Add all the ML operators
        	\item Optimise
        \end{enumerate}
    \item Benchmark
        \begin{enumerate}
        	\item Compare the ML operator to conventional ones
        	\item Compare the Parareal framework with and without ML
        	\item Measure the execution time of the Parareal (ML) framework against a traditional parallel solver
        \end{enumerate}
    \item Write dissertation
\end{enumerate}

\section{Risk Analysis}

\begin{itemize}
    \item Failure to implement a single framework to handle both ODEs and PDEs
    \item Failure to fit a regression model to complex differential equations
    \item Failure to utilise transfer learning
    \item Unavailability of computing resources
    \item Disruptions due to Covid-19
\end{itemize}

\section{Outline of the Dissertation Report}

\begin{enumerate}
	\item Introduction
	\item Background
    	\begin{enumerate}
    		\item Parareal algorithm
    		\item Machine learning for differential equations
    		\item Parareal with neural network based fine operator
    	\end{enumerate}
	\item Problem formulation
    	\begin{enumerate}
    		\item Operator-agnostic Parareal framework
    		\item Advantages of machine learning for coarse operator
    		\item Transfer learning
    	\end{enumerate}
	\item Design and implementation
    	\begin{enumerate}
    	    \item Language and tools
    		\item Parareal framework
            	\begin{enumerate}
            		\item Differential equations
            		\item Integrators
            		\item Operators
            	\end{enumerate}
    		\item Machine learning accelerated operators
    	\end{enumerate}
    \item Results
        \begin{enumerate}
    		\item Machine learning operator performance
    		\item Parareal performance with and without machine learning
    		\item Parareal performance against conventional parallel solver
    	\end{enumerate}
	\item Conclusions
\end{enumerate}

\section{Previous Dissertation Review}

Fitting Large-Scale Gaussian Mixtures With Accelerated Gradient Descent, Nestor Sanchez, 2018.

\pagebreak

\bibliographystyle{unsrt}
\bibliography{ref}

\end{document}

