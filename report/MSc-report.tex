\documentclass{article}

\usepackage{epcc}
\usepackage{graphics}
\usepackage{booktabs}
\usepackage{pgfgantt}

\renewcommand{\labelenumii}{\theenumii}
\renewcommand{\theenumii}{\theenumi.\arabic{enumii}.}


\begin{document}

\pagenumbering{roman}

\title{Parallel-in-time Methods with Machine Learning - Report}
\author{Viktor Csomor}
\date{\today}

\makeEPCCtitle

\newpage

\tableofcontents

\newpage
\pagenumbering{arabic}

\section{Introduction}

Differential equations can model a wide array of dynamical systems that play important roles in science, engineering, and finance. Examples of these include population growth, the motion of fluids, and stock market dynamics. While some simple differential equations can be solved analytically, most of them require a numerical approach \cite[p.~310]{suli2003}. Due to the importance of these models, their accurate and efficient numerical solution is a well researched area of numerical analysis. A lot of this research is focused on the parallelisation of numerical solvers to achieve higher accuracy or reduce execution times. Usually, this parallelisation is applied either across the system or across time \cite{gear1988}. The former generally means exploiting parallelism for the evaluation or integration of the right-hand side of a differential equation at a single point \cite[p.~1-2]{solodushkin2016}. While it is a straightforward approach, it works well only for high dimensional differential equations. On the other hand, parallel-in-time methods allow for the utilisation of high levels of parallelism regardless the dimensionality of the problem.

Our project focuses on the Parareal algorithm \cite{parareal} from the family of parallel-in-time methods. This algorithm divides the time domain up into $P$ slices where $P$ is the number of executing processes or threads.

\section{Background and Literature Review}

To do...

\section{Preliminary Investigations}



\section{Final Proposal}



\section{Workplan}

\begin{enumerate}
    \item Implement framework
        \begin{enumerate}
        	\item Support systems of ODEs
        	\item Support PDEs
        	\item Add ML operators
        	\item Optimise
        \end{enumerate}
    \item Benchmark
        \begin{enumerate}
        	\item ML operators
        	\item Framework with and without ML
        	\item Against another parallel solver
        \end{enumerate}
    \item Write dissertation
        \begin{enumerate}
        	\item Background
        	\item Problem formulation
        	\item Design and implementation
        	\item Results
        	\item Proof read
        \end{enumerate}
\end{enumerate}

\begin{figure}[!htb]
\begin{center}
\ganttset{calendar week text={W\currentweek}}
\begin{ganttchart}[
        hgrid,
        vgrid,
        x unit=1mm,
        y unit chart=5mm,
        time slot format=isodate,
        time slot unit=day]{2020-05-25}{2020-08-23}
    \gantttitlecalendar{month=name, week=1} \\
    \ganttgroup{Implement framework}{2020-05-25}{2020-07-18} \\
    \ganttbar[bar/.style={fill=red!50}]{Support systems of ODEs}{2020-05-25}{2020-06-07} \\
    \ganttbar[bar/.style={fill=red!50}]{Support PDEs}{2020-05-31}{2020-06-14} \\
    \ganttbar[bar/.style={fill=red!50}]{Add ML operators}{2020-06-14}{2020-06-28} \\
    \ganttbar[bar/.style={fill=red!50}]{Optimise}{2020-06-28}{2020-07-18} \\
    \ganttgroup{Benchmark}{2020-07-04}{2020-07-17} \\
    \ganttbar[bar/.style={fill=green!50}]{ML operators}{2020-07-04}{2020-07-10} \\
    \ganttbar[bar/.style={fill=green!50}]{Framework with and without ML}{2020-07-08}{2020-07-13} \\
    \ganttbar[bar/.style={fill=green!50}]{Against another parallel solver}{2020-07-10}{2020-07-17} \\
    \ganttgroup{Write dissertation}{2020-05-25}{2020-08-21} \\
    \ganttbar[bar/.style={fill=blue!50}]{Background}{2020-05-25}{2020-06-11} \\
    \ganttbar[bar/.style={fill=blue!50}]{Problem formulation}{2020-06-12}{2020-06-28} \\
    \ganttbar[bar/.style={fill=blue!50}]{Design and implementation}{2020-06-29}{2020-07-24} \\
    \ganttbar[bar/.style={fill=blue!50}]{Results}{2020-07-25}{2020-08-15} \\
    \ganttbar[bar/.style={fill=blue!50}]{Proof read}{2020-08-16}{2020-08-21}
\end{ganttchart}
\end{center}
\caption{Gantt chart of the dissertation work plan}
\label{fig:gantt}
\end{figure}

\section{Risk Analysis}

We identify the main risks as:

\begin{itemize}
    \item Failure to implement a single framework to handle both ODEs and PDEs
    \item Failure to fit a regression model to complex differential equations
    \item Failure to utilise transfer learning
    \item Unavailability of computing resources
    \item Disruptions due to Covid-19
\end{itemize}

\section{Outline of the Dissertation Report}

\begin{enumerate}
	\item Introduction
	\item Background
    	\begin{enumerate}
    		\item Parareal algorithm
    		\item Machine learning for differential equations
    		\item Parareal with neural network based fine operator
    	\end{enumerate}
	\item Problem formulation
    	\begin{enumerate}
    		\item Operator-agnostic Parareal framework
    		\item Advantages of machine learning for coarse operator
    		\item Transfer learning
    	\end{enumerate}
	\item Design and implementation
    	\begin{enumerate}
    	    \item Language and tools
    		\item Parareal framework
            	\begin{enumerate}
            		\item Differential equations
            		\item Integrators
            		\item Operators
            	\end{enumerate}
    		\item Machine learning accelerated operators
    	\end{enumerate}
    \item Results
        \begin{enumerate}
    		\item Machine learning operator performance
    		\item Parareal performance with and without machine learning
    		\item Framework performance against conventional parallel solvers
    	\end{enumerate}
	\item Conclusions
\end{enumerate}

\section{Previous Dissertation Review}

Fitting Large-Scale Gaussian Mixtures With Accelerated Gradient Descent, Nestor Sanchez, 2018.

\pagebreak

\bibliographystyle{unsrt}
\bibliography{ref}

\end{document}

